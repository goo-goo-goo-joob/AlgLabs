%!TEX TS-program = xelatex

% Шаблон документа LaTeX создан в 2018 году
% Алексеем Подчезерцевым
% В качестве исходных использованы шаблоны
% 	Данилом Фёдоровых (danil@fedorovykh.ru) 
%		https://www.writelatex.com/coursera/latex/5.2.2
%	LaTeX-шаблон для русской кандидатской диссертации и её автореферата.
%		https://github.com/AndreyAkinshin/Russian-Phd-LaTeX-Dissertation-Template

\documentclass[a4paper,14pt]{article}

\input{data/preambular.tex}
\begin{document} % конец преамбулы, начало документа
\input{data/title.tex}
\setcounter{page}{2} % нумерация

\renewcommand\contentsname{\centering {\normalsize Содержание}}
\tableofcontents
\newpage

\section*{Постановка задачи}
\addcontentsline{toc}{section}{Постановка задачи}

Используя алгоритмы стандартной библиотеки \textit{<algorithm>} и, при необходимости, \textit{<functional>}, выполнить задания в соответствии с вариантом 18:
\begin{enumerate}
	\item Ввести произвольную строку с клавиатуры и удалить все пробелы. Ввести слово и, если оно не встречается в ранее введенной строке, вывести сообщение об ошибке.
	\item Написать программу, создающую вектор чисел, и копирующую все числа кратные заданному числу в новый вектор.
	\item На одометре (прибор для измерения пробега автомобиля) число 15951. Через 2 часа езды он показывает другое число палиндром. Какова средняя скорость? Написать программу, позволяющую задать новое значение на одометре (новое число должно быть больше начального (15951) и быть палиндромом) и вычисляющую среднюю скорость.
\end{enumerate}

\newpage

\section{Основная часть}
\subsection{Общая идея решения задачи}
Для решения задачи были использованы:
\begin{enumerate}
	\item Aлгоритмы стандартной библиотеки \textit{<algorithm>}: \textit{remove, search, generate, copy if, reverse copy, for each}.
	\item Контейнеры \textit{<vector>} и \textit{<string>}.
	\item Функция \textit{srand} из библиотеки \textit{<cstdlib>} и функция \textit{time} из библиотеки \textit{<ctime>} для генерации случайных чисел.
	\item Лямбда-выражения.
\end{enumerate}
\subsection{Структура и принципы действия}
\subsubsection{Задание 1}
В функции \textit{task 5} происходит считывание произвольной строки, введенной с клавиатуры. Далее с помощью функции \textit{remove} удаляются все пробелы из строки. Затем функция \textit{erase} позволяет удалить из строки ненужные элементы, появившиеся после исключения всех пробелов. После происходит считывание подстроки, функция \textit{search} проверяет входить ли эта последовательность символов в строку и возвращает итератор к первому элементу вхождения. Если итератор указывает на конец строки, то выводится сообщение об ошибке, то есть подстрока не найдена.

\subsubsection{Задание 2}
В функции \textit{task 11} происходит создание двух векторов одинаковой длины (для наглядности длина 10). Далее инициализируется генератор случайных чисел с помощью функций \textit{srand} и \textit{time}. Затем функция \textit{generate} заполняет вектор случайными значениями (для удобства в диапазоне от 0 до 9), используя функцию-генератор в виде лямбда-выражения. После этого пользователь вводит число, чтобы проверить значения вектора на кратность используется функция \textit{copy if}. Она копирует значения из одного вектора в другой, если параметр-функция, реализованная в виде лямбда-выражения, вернет \textit{true}. Лямбда-выражение проверяет число из вектора на кратность введенному числу.  

\subsubsection{Задание 3}
В функции \textit{task 14} осуществляется генерация случайного вектора-палиндрома. Для этого генератор случайных чисел инициализируется с помощью функций \textit{srand} и \textit{time}. Затем функция \textit{generate} заполняет вектор длины 5 случайными значениями от 0 до 9, используя функцию-генератор в виде лямбда-выражения. Далее функция \textit{reverse copy} отображает вектор в обратном порядке. Операции генерации и отображения повторяются, пока не будет создан вектор-палиндром, больший заданного в условии вектора. После того, как вектор будет получен, с помощью функции \textit{for each} и лямбда-выражения вектор переводится в число, и затем считается и выводится на экран средняя скорость.

\subsection{Процедура получения исполняемых программных модулей}
Программный код был скомпилирован с среде \textit{Visual Studio 2017}. Код программы содержится в одном исходном файле. Для ускоренной компиляции программы используются предварительно откомпилированные заголовки \textit{"pch.h"}. Помимо этого никаких дополнительных ключей не добавлялось, использовались ключи, которые добавляются по умолчанию.
\subsection{Результаты тестирования}
Тестирование программы представлено в файле \textit{"Task5.cpp"} в функции \textit{Main()}. Ожидаемый вывод функции:
\begin{verbatim}
your string:microsoft visual studio top
new string:microsoftvisualstudiotop
your substring:nottop
Can't find
	
random vector: 7 4 2 7 0 8 5 2 9 6
your number: 7
new vector: 7 7 0
	
new value: 95859
speed: 39954
\end{verbatim}

\newpage
\setcounter{figure}{1} 
\setcounter{section}{1} 
\setcounter{subsection}{1} 

\begin{center}
	\section*{Приложение А}
	полный код программы
	\addcontentsline{toc}{section}{Приложение А}
	
\end{center}

\renewcommand{\subsection}{\Asbuk{section}.\arabic{subsection}}
\setcounter{subsection}{1} 
\textbf{\subsection{  - Task5.cpp}}
\addcontentsline{toc}{subsection}{Task5.cpp}
\lstinputlisting[language=C++]{../Task5/Task5.cpp}


\end{document} % конец документа
