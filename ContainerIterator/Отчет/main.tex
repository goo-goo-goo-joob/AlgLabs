%!TEX TS-program = xelatex

% Шаблон документа LaTeX создан в 2018 году
% Алексеем Подчезерцевым
% В качестве исходных использованы шаблоны
% 	Данилом Фёдоровых (danil@fedorovykh.ru) 
%		https://www.writelatex.com/coursera/latex/5.2.2
%	LaTeX-шаблон для русской кандидатской диссертации и её автореферата.
%		https://github.com/AndreyAkinshin/Russian-Phd-LaTeX-Dissertation-Template

\documentclass[a4paper,14pt]{article}

\input{data/preambular.tex}
\begin{document} % конец преамбулы, начало документа
\input{data/title.tex}
\setcounter{page}{2} % нумерация

\renewcommand\contentsname{\centering {\normalsize Содержание}}
\tableofcontents
\newpage

\section*{Постановка задачи}
\addcontentsline{toc}{section}{Постановка задачи}

Используя механизм шаблонных классов (\textit{template}) написать классы для динамических структур данных и итераторов для их обхода в соответствии с вариантом 17:
\begin{enumerate}
	\item Бинарное дерево - \textit{ForwardIterator} с условием
	\item Очередь - \textit{ForwardIterator}
\end{enumerate}

\newpage

\section{Основная часть}
\subsection{Общая идея решения задачи}
Для решения задачи были созданы шаблонные классы \textit{Queue} и \textit{BinaryTree} с помощью агрегирования классов итераторов (\textit{Iterator} и \textit{ConstIterator}, и \textit{TreeIterator} и \textit{ConstTreeIterator} соответственно), классы итераторов в свою очередь содержат структуру узла \textit{Node} или \textit{TreeNode} соответственно.
\subsection{Структура и принципы действия}
Классы \textit{Queue}, \textit{Iterator} и \textit{ConstIterator} содержат в себе структуру узла (\textit{Node}), который состоит из элемента очереди и указателя на следующий элемент. 

Классы \textit{BinaryTree}, \textit{TreeIterator} и \textit{ConstTreeIterator} содержат в себе структуру узла (\textit{TreeNode}), который состоит из элемента бинарного дерева и указателей на родительский, левый и правый элементы. 

Класс \textit{Queue} содержит в себе указатель на начало очереди и конец очереди (\textit{Node*}) и размер, определены конструктор умолчания, конструктор копирования, деструктор, а также такие стандартные методы, как \textit{begin} (возвращает указатель на первый элемент),\textit{end} (возвращает указатель на последний элемент), \textit{len} (возвращает размер очереди), \textit{push} (добавляет элемент в конец очереди), \textit{pop} (удаляет элемент из начала очереди и возвращает его).

Класс \textit{BinaryTree} содержит в себе указатель на начальную вершину бинарного дерева (\textit{TreeNode*}), указатель на функцию, которая задает условие добавления элемента в бинарное дерево, определены конструктор с параметром (принимает указатель на функцию), конструктор копирования, деструктор, а также такие стандартные методы, как \textit{begin} (возвращает указатель на вершину дерева),\textit{end} (возвращает нулевой указатель), \textit{push} (добавляет элемент в бинарное дерево в соответствии с условием).

Каждый класс содержит классы \textit{Iterator} и \textit{ConstIterator}, или \textit{TreeIterator} и\\ \textit{ConstTreeIterator}, которые содержат указатель на элемент структуры \textit{Node} или \textit{TreeNode}, конструктор с параметром, для которых определены операторы \textit{++}, \textit{*},\textit{ ==}, \textit{!=}.

\begin{figure}[H]
	\centering
	\includegraphics[width=\linewidth]{queue.png}
	\caption{Диграма классов \textit{Queue}}	
\end{figure}

\begin{figure}[H]
	\centering
	\includegraphics[width=\linewidth]{binarytree.png}
	\caption{Диграма классов \textit{BinaryTree}}	
\end{figure}

Также были реализованы шаблонные функции, перегружающие оператор вывода в поток для отображения коллекций \textit{Queue} и \textit{BinaryTree} на экране.

\subsection{Процедура получения исполняемых программных модулей}
Программный код был скомпилирован с среде \textit{Visual Studio 2017}. Компиляция раздельная: исходный код программы разделён на несколько файлов. Для ускоренной компиляции программы используются предварительно откомпилированные заголовки \textit{"pch.h"}. Помимо этого никаких дополнительных ключей не добавлялось, использовались ключи, которые добавляются по умолчанию. 
\subsection{Результаты тестирования}
Тестирование программы представлено в файле \textit{"ContainerIterator.cpp"} в функции \textit{Main()}. Ожидаемый вывод функции:
\begin{verbatim}
1 2 -2 4 -4 5 -5 6 6 -6 123 234
1 2 -2 4 -4 5 -5 6 6 -6 123 234

1.7 0.333
0.333
1
\end{verbatim}

\newpage
\setcounter{figure}{1} 
\setcounter{section}{1} 
\setcounter{subsection}{1} 

\begin{center}
	\section*{Приложение А}
	полный код программы
	\addcontentsline{toc}{section}{Приложение А}
	
\end{center}


\renewcommand{\subsection}{\Asbuk{section}.\arabic{subsection}}
\setcounter{subsection}{1} 
\textbf{\subsection{  - BinaryTree.h}}
\addcontentsline{toc}{subsection}{BinaryTree.h}
\lstinputlisting[language=C++]{../ContainerIterator/BinaryTree.h}


\setcounter{subsection}{2} 
\textbf{\subsection{  - Queue.h}}
\addcontentsline{toc}{subsection}{Queue.h}
\lstinputlisting[language=C++]{../ContainerIterator/Queue.h}


\setcounter{subsection}{3} 
\textbf{\subsection{  - Node.h}}
\addcontentsline{toc}{subsection}{Node.h}
\lstinputlisting[language=C++]{../ContainerIterator/Node.h}


\setcounter{subsection}{4} 
\textbf{\subsection{  - Iterator.h}}
\addcontentsline{toc}{subsection}{Iterator.h}
\lstinputlisting[language=C++]{../ContainerIterator/Iterator.h}


\setcounter{subsection}{5} 
\textbf{\subsection{  - ContainerIterator.cpp}}
\addcontentsline{toc}{subsection}{ContainerIterator.h}
\lstinputlisting[language=C++]{../ContainerIterator/ContainerIterator.cpp}
\end{document} % конец документа
