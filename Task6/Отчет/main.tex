%!TEX TS-program = xelatex

% Шаблон документа LaTeX создан в 2018 году
% Алексеем Подчезерцевым
% В качестве исходных использованы шаблоны
% 	Данилом Фёдоровых (danil@fedorovykh.ru) 
%		https://www.writelatex.com/coursera/latex/5.2.2
%	LaTeX-шаблон для русской кандидатской диссертации и её автореферата.
%		https://github.com/AndreyAkinshin/Russian-Phd-LaTeX-Dissertation-Template

\documentclass[a4paper,14pt]{article}

\input{data/preambular.tex}
\begin{document} % конец преамбулы, начало документа
\input{data/title.tex}
\setcounter{page}{2} % нумерация

\renewcommand\contentsname{\centering {\normalsize Содержание}}
\tableofcontents
\newpage

\section*{Постановка задачи}
\addcontentsline{toc}{section}{Постановка задачи}

Использовать классы \textit{thread}, \textit{mutex} (или более подходящий подкласс), \textit{lock\_guard} / \textit{unique\_lock}, \textit{condition\_variable}, \textit{atomic}, \textit{future} и \textit{async} при решении заданий (в зависимости от условий задания).
Программа дуэль: есть секундант (отдельный поток), отсчитывающий время, и есть два дуэлянта (еще два потока), получающих сигнал о выстреле независимо (параллельно), реализовать выстрелы: первого стрелка – «первый», второго – «второй» и реализовать механизм в соответствии с вариантом 18 (3):
один стрелок реагирует мгновенно, стреляет с задержкой, второй наоборот, реализовать таймеры или иной механизм иллюстрирующий эти задержки.

\newpage

\section{Основная часть}
\subsection{Общая идея решения задачи}
Для решения задачи были использованы классы \textit{thread}, \textit{condition\_variable} и \textit{mutex} (\textit{unique\_lock}).
\subsection{Структура и принципы действия}
Программа содержит глобальные переменные \textit{mutex} \textit{lock}, \textit{condition\_variable} \textit{cv} и \textit{bool} \textit{ready}.

В начале программы инициализируется генератор случайных чисел с помощью функций \textit{srand} и \textit{time}. Далее создаются 3 новых объекта потока: секундант (\textit{sec}) и два дуэлянта (\textit{d1} и \textit{d2}), и связываются с потоком выполнения, новый поток выполнения вызывает, соответственно, функции \textit{go}, \textit{threadFunction1} и \textit{threadFunction2}.

Функция \textit{go} для защиты от одновременного доступа нескольких потоков создает экземпляр класса \textit{unique\_lock}, который принимает не захваченный \textit{mutex} \textit{lock} в конструкторе. Далее выводится сообщение о готовности, с помощью метода \textit{sleep\_for} имитируется задержка секунданта перед стартом, глобальная переменная \textit{ready} меняет свое значение на \textit{true} и подается сигнал для старта. После чего \textit{condition\_variable} \textit{cv} уведомляет все ожидающие потоки.

Функция \textit{threadFunction1} принимает сигнал от \textit{cv} мгновенно выводит сообщение о получении сигнала, имитирует задержку с помощью метода \textit{sleep\_for} и выводит сообщение о выстреле.

Функция \textit{threadFunction2} принимает сигнал от \textit{cv} имитирует задержку перед получением сигнала с помощью метода \textit{sleep\_for} и выводит сообщение о получении сигнала и выстреле.

\subsection{Процедура получения исполняемых программных модулей}
Программный код был скомпилирован с среде \textit{Visual Studio 2017}. Код программы содержится в одном исходном файле. Для ускоренной компиляции программы используются предварительно откомпилированные заголовки \textit{"pch.h"}. Помимо этого никаких дополнительных ключей не добавлялось, использовались ключи, которые добавляются по умолчанию.
\subsection{Результаты тестирования}
Тестирование программы представлено в файле \textit{"Task6.cpp"} в функции \textit{Main()}. Ожидаемый вывод функции:
\begin{verbatim}
Ready..
Go!
Pushkin react
Dantes react
Dantes shoot
Pushkin shoot
\end{verbatim}

\newpage
\setcounter{figure}{1} 
\setcounter{section}{1} 
\setcounter{subsection}{1} 

\begin{center}
	\section*{Приложение А}
	полный код программы
	\addcontentsline{toc}{section}{Приложение А}
	
	\renewcommand{\subsection}{\Asbuk{section}.\arabic{subsection}}
	\setcounter{subsection}{1} 
	\textbf{\subsection{  - Task6.cpp}}
	\addcontentsline{toc}{subsection}{Task6.cpp}
	\lstinputlisting[language=C++]{../Task6/Task6.cpp}
	
\end{center}



\end{document} % конец документа
